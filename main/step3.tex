\subsection{草图第三阶段}
在本节中,长度的单位是mm。
\subsubsection{减速器机体的结构设计}
\begin{itemize}
	\item [a)]机体中心高和油面位置的确定
	\par 为防止浸油齿轮将油池底部沉积物搅起,大齿轮的齿顶圆到油池底面的距离应不小于$30\sim 50$。应保证齿轮浸入深度应不小于10,最高油面应比最低油面高出$10\sim 15$,且齿轮浸油深度最多不超过齿轮半径的$\dfrac{1}{4} \sim \dfrac{1}{3}$。按照以上原则,选择机体中心高H=198 ,油面高度为120,满足以上要求。
    \par 验算油量:单级减速器传递1kW功率需要油量为$0.35\sim 0.7\mathrm{dm^3}$,本设计为二级齿轮减速器,传递的需油量$$V_0=2\times \left(0.35\sim 0.7\right)\times 1.34\mathrm{kW}=\left(0.938\sim 1.876\right)\times 10^6~\mathrm{mm^3}$$
    油池容量$$V=104\times 470\times 141~\mathrm{mm^3}=6.892\times 10^6~\mathrm{mm^3}$$
    $V>V_0$,满足设计,润滑条件较好。
    
    \item[b)] 其他结构
    \par 其他结构设计详见A0图纸。
\end{itemize}
\subsubsection{减速器的附件设计}
\begin{itemize}
	\item [a)]窥视孔盖和窥视孔
	\par 在机盖顶部开有窥视孔,能看到 传动零件齿合区的位置,并有足够的空间,以便于能伸入进行操作,窥视孔有盖板,机体上开窥视孔与铸造的凸缘一块,有便于机械加工出支承盖板的表面并用垫片加强密封,盖板用钢板焊接制成,用M6螺栓紧固。
	\par 按要求选取$D=22$,$D_1=19.6$,$L=23$,$l=12$,$a=2$,$d_1=5$,螺钉尺寸$\mathrm{M}6\times 16$螺钉数目为4,具体尺寸见参考文献[3]19页。
	\item[b)]放油螺塞
	\par 放油孔位于油池最底处,并安排在减速器中部,以便放油,放油孔用螺塞堵住,并加皮油封垫圈加以密封。选用六角螺塞M18(JB/ZQ 4450-1986)。
	\item[c)]油标指示器
	\par 选取M12的杆式油标。参数如下:
	\begin{table}[H]
		\begin{center}
			\caption{M12的杆式油标相关参数}
			\begin{tabular}{cccccccccc}
				\toprule
				$d$& $d_1$ & $d_2$ & $d_3$ & $h$ & $a$ & $b$ & $c$ & $D$ & $D_1$ \\
				\midrule
				M12&	4 &	12 &	6 &	28 &	10 &	6 & 4 &20& 16 \\
				\bottomrule
			\end{tabular}
		\end{center}
	\end{table}
	\par 具体尺寸见参考文献[3]19页。油标位置箱体中部。油尺安置的部位不能太低,以防油进入油尺座孔而溢出.
	\item[d)] 通气孔
	\par 由于减速器运转时,机体内温度升高,气压增大,为便于排气,在机盖顶部的窥视孔改上安装通气器,以便达到体内为压力平衡。由于是在清洁无尘的环境下,只需使用简易通气孔。选取简易通气孔。具体尺寸选取查阅参考文献[3]P19页。
	\item[e)] 吊钩和吊耳
	\par 在机盖上上直接铸出吊耳和吊钩,用以起吊或搬运较重的物体。吊耳参数如下:$d=\left(1.8\sim 2.5\right)\delta_1=(14.4\sim 20)~\mathrm{mm}$,取$d=20\mathrm{mm}$;$e=\left(0.8\sim 1.0\right)d=\left(16\sim 20\right)$,取$e=20~\mathrm{mm}$;$s=2\delta_1=16~\mathrm{mm}$;取$R=20~\mathrm{mm}$。
	\par 吊钩参数如下:$B=c_1+c_2=16+14=30~\mathrm{mm}$,$H=\left(0.8\sim 1.2\right)B=(24~30)~\mathrm{mm}$,取$H=30~\mathrm{mm}$;$\ h=0.5H=0.5\times 30=15~\mathrm{mm}$,$s=2\delta=16~\mathrm{mm}$;$r=0.25B=7.5~\mathrm{mm}$。 
	\par 具体尺寸由参考文献[3]20页的经验公式选取。
	\item[f)] 定位销:
	\par 为保证剖分式机体的轴承座孔的加工及装配精度,在机体联结凸缘的长度方向各安装一圆锥定位销,以提高定位精度。选取公称直径为10的圆锥销,采用非对称布置。具体尺寸见参考文献[1]表11.30圆锥销(GB/T117-2000)。
	\item[g)] 启盖螺钉
	\par 启盖螺钉上的螺纹长度要大于机盖联结凸缘的厚度。选取与机盖和机座连接螺栓相同规格的螺栓作为启盖螺栓。螺钉杆端部要做成圆柱形或大倒角,以免破坏螺纹。
\end{itemize}






