这里我们的工作主要集中在齿轮的选择上。这里我们不妨将\uppercase\expandafter{\romannumeral1}轴上的主动轮叫做齿轮a,将\uppercase\expandafter{\romannumeral2}轴上的从动轮叫做齿轮b,将\uppercase\expandafter{\romannumeral2}轴上的主动轮叫做齿轮c,将\uppercase\expandafter{\romannumeral3}轴上的从动轮叫做齿轮d。
\par 本节扭矩的单位是$\mathrm{N}\cdot \mathrm{mm}$,长度的单位为mm,应力的单位为MPa
\subsection{选择齿轮材料、热处理方式和精度等级}
考虑到带式运输机的结构紧凑性,四个齿轮均采用40Cr,采用软齿面。两个主动齿轮调质处理,两个从动齿轮正火处理,选用8级精度。
\subsection{齿轮相关参数的选择}
\begin{itemize}
	\item[a)] 四个齿轮的齿数
	\par 经过反复计算,我们选择了如下参数:$z_a=19$,$z_b=93$,$z_c=29$,$z_d=99$,满足$i_{\mathrm{\uppercase\expandafter{\romannumeral1}}}=4.9$,$i_{\mathrm{\uppercase\expandafter{\romannumeral2}}}=3.41$。
	\item[b)] 四个齿轮的齿宽系数的确定
	\par 参考\cite{2},我们得到齿宽系数$\phi_d=1$。
	\item[c)] 载荷系数的确定
	\par 机器在运行的过程中受到轻微冲击,又是电机驱动,故根据\cite{2}可知,载荷系数$K=1.2$
	\item[d)] 齿数比$u_{ab}$和$u_{cd}$
	\par 经过计算,$$u_{ab}=\frac{z_b}{z_a}=4.92$$	$$u_{cd}=\frac{z_d}{z_c}=3.41$$
\end{itemize}
\subsection{按照齿面接触疲劳强度设计齿轮并确定其它参数}
\begin{itemize}
	\item [a)]齿轮a和齿轮b
	$$T_1=9550\frac{P_{\mathrm{\uppercase\expandafter{\romannumeral1}}}}{n_\mathrm{\uppercase\expandafter{\romannumeral1}}}=1.704\times 10^4$$
	$$\left[\sigma_H\right]=\min\left\{\left[\sigma_H\right]_{a,c},~\left[\sigma_H\right]_{b,d}\right\}=700$$
	a的分度圆直径
	$$d_{a}\geq 76.6\sqrt[3]{\frac{KT_{1}\left(u_{ab}+1\right)}{\phi_{d}u_{ab}\left[\sigma_H\right]^2}}=28.26$$
	\par a、b齿轮的模数a、b齿轮的分度圆直径
	经过计算,$m_{ab}\geq \dfrac{d_a}{z_a}\approx 1.48$,为了取整令$m_{ab}=1.5$。由此得出$d_a=28.5$,$d_b=z_bm_{ab}=139.5$。
	\par a和b齿轮的中心距
	$$a_{ab}=\dfrac{m_{ab}\left(z_a+z_b\right)}{2}=84$$
	\par a和b齿轮的齿宽
	\par a的齿宽$b_a=\phi_{d}d_a=28.5$,故最终确定$b_b=25$,$b_a=30$。
	\item [b)]齿轮c和齿轮d
	$$T_1=9550\frac{P_{\mathrm{\uppercase\expandafter{\romannumeral2}}}}{n_\mathrm{\uppercase\expandafter{\romannumeral2}}}=7.939\times 10^4$$
	$$\left[\sigma_H\right]=\min\left\{\left[\sigma_H\right]_{a,c},~\left[\sigma_H\right]_{b,d}\right\}=700$$
	a的分度圆直径
	$$d_{c}\geq 76.6\sqrt[3]{\frac{KT_{2}\left(u_{cd}+1\right)}{\phi_{d}u_{cd}\left[\sigma_H\right]^2}}=28.95$$
	\par c、d齿轮的模数和c、d齿轮的分度圆直径
	经过计算,$m_{ab}\geq \dfrac{d_a}{z_a}\approx 1.43$,由于是输出轴,动力较大,为了防止过载,令$m_{cd}=2$。由此得出$d_c=62$,$d_d=z_dm_{cd}=202$。
	\par c和d齿轮的中心距
	$$a_{cd}=\dfrac{m_{cd}\left(z_c+z_d\right)}{2}=128$$
	\par c和d齿轮的齿宽
	\par c的齿宽$b_c=\phi_{d}d_c=58$,故最终确定$b_c=60$,$b_d=55$。
\end{itemize}
\subsection{校核齿根弯曲疲劳强度}
\begin{itemize}
	\item[a)] 齿轮a和齿轮b
	\par 由\cite{2},得到$\sigma_{\mathrm{Flim}a,c}=220$,$\sigma_{\mathrm{Flim}b,d}=170$,
	$\left[\sigma\right]_F=\dfrac{Y_N\sigma_{\mathrm{Flim}}}{S_F}$,查得安全系数$S_F=1.25$,$Y_{Fa}=1.97$,$Y_{Fb}=2.22$。
	$\sigma_{Fb}=\dfrac{2KT_1Y_{Fb}}{b_bd_bm_{ab}}=4.21\ll \left[\sigma\right]_{Fb,d}$ 
	\par 在强度许用范围内,故齿轮a和齿轮b能够安全工作。
	\item[b)] 齿轮c和齿轮d
	\par 由\cite{2},得到$\sigma_{\mathrm{Flim}a,c}=220$,$\sigma_{\mathrm{Flim}b,d}=170$,
	$\left[\sigma\right]_F=\dfrac{Y_N\sigma_{\mathrm{Flim}}}{S_F}$,查得安全系数$S_F=1.25$,$Y_{Fc}=1.97$,$Y_{Fd}=2.22$。
	$\sigma_{Fd}=\dfrac{2KT_1Y_{Fd}}{b_dd_dm_{cd}}=4.21\ll \left[\sigma\right]_{Fb,d}$ 
	\par 在强度许用范围内,故齿轮c和齿轮d能够安全工作。
\end{itemize}
综上所述,最终四个齿轮的参数列于下表:
\begin{table}[H]
	\begin{center}
		\caption{齿轮a、b、c、d的相关参数}
		\begin{tabular}{ccccccc}
			\toprule
			齿轮序号 & 模数$m$ & 齿数$z$ & 分度圆直径$d$ & 齿顶圆直径$d_a$ & 齿根圆直径$d_f$ & 中心距$a$ \\
			\midrule
			a & \multirow{2}{*}{1.5} & 19 & 28.5 & 31.5 & 14.75 & \multirow{2}{*}{84} \\
			b & ~ & 93 & 139.5 & 142.5 & 135.75 & ~ \\
			c & \multirow{2}{*}{2} & 29 & 58 & 62 & 57 & \multirow{2}{*}{128} \\
			d & ~ & 99 & 198 & 202 & 193 & ~ \\
			\bottomrule
		\end{tabular}
	\end{center}
\end{table}