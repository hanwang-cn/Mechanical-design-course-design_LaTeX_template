在该小节中,功率的单位是$\mathrm{kW}$,扭矩的单位是$\mathrm{N\cdot m}$,转速的单位是$\mathrm{r\cdot min^{-1}}$。
\subsection{电动机的选择}
\subsubsection{选择电动机的类型}
根据设计要求和工作条件选用Y系列三相笼型异步电动机,
其结构为全封闭自扇冷式结构,电压为380 V。
\subsubsection{选择电动机的容量}
根据设计数据,工作机的有效功率为
$$P_{W}=\frac{Fv}{1000}=\frac{1900\times{1.1}}{1000}=2.09$$
从电动机到工作机输送带之间的总效率为:
$$\eta_{\Sigma}={{\eta_1}^{2}}\cdot{{\eta_2}^{4}}\cdot{{\eta_3}^{2}}\cdot{\eta_4}$$
式中,\\
$\eta_1$——联轴器效率; \\
$\eta_2$——滚动轴承效率; \\
$\eta_3$——齿轮传动效率; \\
$\eta_4$——卷筒的传递效率。 \\
\par 由参考文献\cite{1}的表9.1取$\eta_1=0.99$、$\eta_2=0.98$、$\eta_3=0.97$、$\eta_4=0.96$,则
$$\eta_{\Sigma}={{\eta_1}^{2}}\cdot{{\eta_2}^{4}}\cdot{{\eta_3}^{2}}\cdot{\eta_4}=0.817$$
所以电动机所需工作功率为
$$P_d=\frac{P_W}{\eta_{\Sigma}}=2.56$$
\subsubsection{确定电动机转速}
按参考文献\cite{1}表9.2推荐的传动比合理范围,二级圆柱齿轮减速器传动比${i_{\Sigma}}^\prime=8\sim 40$,而工作机卷筒轴的转速为
$$n_W=\frac{60\times 1000v}{\pi d}=\frac{60\times 1000\times 0.85}{\pi\times 250}\approx 65$$
所以电动机转速的可选范围为
$$n_d={i_\Sigma}n_W=\left(8\sim 40\right)\times 85=680\sim 3400$$
\par 符合这一范围的同步转速有750、1000、1500和3000四种。综合考虑电动机和传动装置的尺寸、质量、及价格等因素,为使传动装置结构紧凑,决定选用同步转速为1500 的电动机。
\par 根据电动机类型、容量和转速,查参考文献\cite{1}表15.1选定电动型号为Y100L2-4,其主要性能如下表:
\begin{table}[H]
	\begin{center}
		\caption{Y100L2-4型电动机的主要性能}
		\begin{tabular}{ccccc}
		\toprule
		电动机型号& 额定功率 &  满载转速&  起动转矩/额定转矩 &最大转矩/额定转矩 \\
		\midrule
		Y100L2-4& 3.2 & 1420 & 2.2 & 2.2 \\
		\bottomrule
		\end{tabular}
	\end{center}
\end{table}
\begin{table}[H]
	\begin{center}
		\caption{ Y100L2-4型电动机的主要性能}
		\begin{tabular}{ccccccccc}
			\toprule
			型号 & $H$ & $A$ & $B$ & $C$ & $D$ & $E$ & $F\times GD$ & $G$\\
			\midrule
			Y100L& 100 & 160 & 140 & 63 & 28 & 60 & $8\times 7$ &24 \\
			\midrule[0.4mm]
			  $K$ & $b$ & $b_1$ & $b_2$ & $h$ & $AA$ & $BB$ & $HA$ & $L_1$ \\
			 \midrule
			   12 & 205 & 180 & 105 & 245 & 40 & 176 & 14 & 380\\
			\bottomrule
		\end{tabular}
	\end{center}
\end{table}
\subsection{计算传动装置总传动比并分配传动比}
总传动比$i_\Sigma$为
$$i_{\Sigma}=\frac{n_M}{n_W}=\frac{940}{85}=16.71$$
分配传动比
$$i_{\Sigma}=i_{\mathrm{\uppercase\expandafter{\romannumeral1}}}i_{\mathrm{\uppercase\expandafter{\romannumeral2}}}$$
\par 考虑润滑条件,为使结构紧凑,各级传动比均在推荐值范围内,取$i_{\mathrm{\uppercase\expandafter{\romannumeral1}}}=1.4i_{\mathrm{\uppercase\expandafter{\romannumeral2}}}$,故
$$i_{\mathrm{\uppercase\expandafter{\romannumeral1}}}=\sqrt{1.4i_{\Sigma}}=4.9$$
$$i_{\mathrm{\uppercase\expandafter{\romannumeral2}}}=\frac{i_\Sigma}{i_{\mathrm{\uppercase\expandafter{\romannumeral1}}}}=3.41$$
\subsection{计算传动装置各轴的运动及动力参数}
\subsubsection{各轴的转速}
\par \uppercase\expandafter{\romannumeral1}轴:
$$n_{\mathrm{\uppercase\expandafter{\romannumeral1}}}=n_M=1420$$
\par \uppercase\expandafter{\romannumeral2}轴:  
$$n_{\mathrm{\uppercase\expandafter{\romannumeral2}}}=\frac{n_{\mathrm{\uppercase\expandafter{\romannumeral1}}}}{i_{\mathrm{\uppercase\expandafter{\romannumeral1}}}}=289.8$$
\par \uppercase\expandafter{\romannumeral3}轴:  
$$n_{\mathrm{\uppercase\expandafter{\romannumeral3}}}=\frac{n_{\mathrm{\uppercase\expandafter{\romannumeral2}}}}{i_{\mathrm{\uppercase\expandafter{\romannumeral1}}}}=85$$
\par 卷筒轴:$$n_{卷筒}=n_{\mathrm{\uppercase\expandafter{\romannumeral3}}}=85$$ 
\subsubsection{各轴的输入功率}
\par \uppercase\expandafter{\romannumeral1}轴:   
$$P_{\mathrm{\uppercase\expandafter{\romannumeral1}}}=P_{d}\eta_{1}=2.5344$$
\par \uppercase\expandafter{\romannumeral2}轴:  
$$P_{\mathrm{\uppercase\expandafter{\romannumeral2}}}=P_{\mathrm{\uppercase\expandafter{\romannumeral1}}}\eta_{2}\eta_{3}=2.41$$
\par \uppercase\expandafter{\romannumeral3}轴: 
$$P_{\mathrm{\uppercase\expandafter{\romannumeral3}}}=P_{\mathrm{\uppercase\expandafter{\romannumeral2}}}\eta_{2}\eta_{3}=2.29$$ 
\par 卷筒轴: 
$$P_{卷}=P_{\mathrm{\uppercase\expandafter{\romannumeral3}}}\eta_{2}\eta_{1}=2.22$$
\subsubsection{各轴的输出转矩}
\par 电动机的输出转矩:
$$T_d=9550\frac{P_d}{n_M}=17.22$$
\par \uppercase\expandafter{\romannumeral1}轴:   
$$T_{\mathrm{\uppercase\expandafter{\romannumeral1}}}=T_{d}\eta_{1}=17.04$$
\par \uppercase\expandafter{\romannumeral2}轴:  
$$T_{\mathrm{\uppercase\expandafter{\romannumeral2}}}=T_{\mathrm{\uppercase\expandafter{\romannumeral1}}}\eta_{2}\eta_{3}i_{\mathrm{\uppercase\expandafter{\romannumeral1}}}=79.39$$
\par \uppercase\expandafter{\romannumeral3}轴: 
$$T_{\mathrm{\uppercase\expandafter{\romannumeral3}}}=T_{\mathrm{\uppercase\expandafter{\romannumeral2}}}\eta_{2}\eta_{3}i_{\mathrm{\uppercase\expandafter{\romannumeral2}}}=257.36$$
\par 卷筒轴: 
$$T_{卷}=T_{\mathrm{\uppercase\expandafter{\romannumeral3}}}\eta_{2}\eta_{1}=249.69$$
\par 将以上结果汇总到表,如下:
\begin{table}[H]
	\begin{center}
	\caption{带传动装置动力和运动数据}
	\begin{tabular}{cccccc}
		\toprule
		轴名 & 功率 & 转矩 & 转速 & 传动比 & $\eta$ \\
		\midrule
		电机轴 & 2.56 & 17.32 & 1420 & —— & 0.99 \\
		\uppercase\expandafter{\romannumeral1}轴 & 2.5344 & 17.04 & 1420 & 1 & 0.99 \\
		\uppercase\expandafter{\romannumeral2}轴 & 2.41 & 29.39 & 289.8 & 4.9 & 0.95 \\
		\uppercase\expandafter{\romannumeral3}轴 & 2.29 & 257.30 & 85 & 3.41 & 0.95 \\
		卷筒轴 & 2.22 & 249.69 & 85 & 1 & 0.97 \\
		\bottomrule
	\end{tabular}
	\end{center}
\end{table}